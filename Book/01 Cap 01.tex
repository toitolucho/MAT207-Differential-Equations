
\section{Definición de una Ecuación Diferencial}
Una ecuación diferencial es una ecuación que involucra una función desconocida y sus derivadas. Su importancia radica en que modelan diversos fenómenos en física, biología, economía e ingeniería.

\subsection*{Ejemplo 1.1: Ecuación Diferencial Básica}
Considere la ecuación diferencial:

\begin{equation}
\frac{dy}{dx} = 3x^2
\end{equation}

Para resolverla, integramos ambos lados:

\begin{equation}
y = \int 3x^2 dx = x^3 + C
\end{equation}

donde \( C \) es la constante de integración.

\section{Clasificación de las Ecuaciones Diferenciales}
Las ecuaciones diferenciales se clasifican según distintos criterios:

\subsection{Ecuaciones Diferenciales Ordinarias (EDO)}
Cuando una ecuación involucra una función de una sola variable independiente y sus derivadas.

\subsection{Ecuaciones Diferenciales Parciales (EDP)}
Si la ecuación involucra derivadas parciales de una función con respecto a más de una variable independiente.

\subsection*{Ejemplo 1.2: EDO vs. EDP}
\begin{itemize}
    \item EDO: \( \frac{d^2y}{dx^2} + 4\frac{dy}{dx} + y = 0 \)
    \item EDP: \( \frac{\partial^2 u}{\partial x^2} + \frac{\partial^2 u}{\partial y^2} = 0 \) (Ecuación de Laplace)
\end{itemize}

\section{Orden y Grado de una Ecuación Diferencial}
El \textbf{orden} de una ecuación diferencial es el de la derivada más alta presente en la ecuación.  
El \textbf{grado} es el exponente de la derivada de orden más alto (si está escrita en forma polinómica).

\subsection*{Ejemplo 1.3: Determinación del Orden y Grado}
Dada la ecuación:

\begin{equation}
\left( \frac{d^2y}{dx^2} \right)^3 + 4\frac{dy}{dx} + y = 0
\end{equation}

\begin{itemize}
    \item \textbf{Orden}: 2 (porque la derivada más alta es \( \frac{d^2y}{dx^2} \))
    \item \textbf{Grado}: 3 (porque la derivada de orden 2 está elevada al cubo)
\end{itemize}

\section{Soluciones de una Ecuación Diferencial}
Existen dos tipos principales de soluciones:

\subsection{Solución General}
Contiene una familia de soluciones dependientes de constantes arbitrarias.

\subsubsection*{Ejemplo 1.4}
Resolver \( \frac{dy}{dx} = 2x \):

\begin{equation}
y = \int 2x dx = x^2 + C
\end{equation}

\subsection{Solución Particular}
Se obtiene al asignar valores específicos a las constantes.

Si se da la condición inicial \( y(1) = 5 \):

\begin{equation}
5 = 1^2 + C \Rightarrow C = 4
\end{equation}

Entonces, la solución particular es \( y = x^2 + 4 \).

\section{Ejercicios}
Resuelve los siguientes ejercicios determinando el orden y grado de cada ecuación:

\begin{enumerate}
    \item \( \frac{d^2y}{dx^2} + 5\frac{dy}{dx} + 6y = 0 \)
    \item \( \left( \frac{dy}{dx} \right)^2 + y = x \)
    \item \( \frac{d^3y}{dx^3} + 2\frac{d^2y}{dx^2} + y = 0 \)
    \item \( \left( \frac{d^2y}{dx^2} \right)^{5} + \frac{dy}{dx} = x^3 \)
    \item \( \frac{\partial u}{\partial x} + \frac{\partial u}{\partial y} = 0 \) (Indica si es EDO o EDP)
    \item \( \frac{dy}{dx} = e^x \)
    \item \( \frac{d^2y}{dx^2} + y^2 = 0 \)
    \item \( x^2\frac{d^3y}{dx^3} + x\frac{d^2y}{dx^2} + y = 0 \)
    \item \( \left( \frac{d^2y}{dx^2} \right) + 3 \frac{dy}{dx} + y = x^2 + 1 \)
    \item \( \left( \frac{d^3y}{dx^3} \right)^2 + y = 0 \)
\end{enumerate}

