\section{Introducción}
Las ecuaciones diferenciales de primer orden son aquellas en las que la derivada de mayor orden es la primera derivada de la función incógnita. Estas ecuaciones aparecen en una gran variedad de aplicaciones en la física, biología, economía e ingeniería.

\section{Ecuaciones de Variables Separables}
Una ecuación diferencial de primer orden se dice separable si puede escribirse en la forma:

\begin{equation}
\frac{dy}{dx} = g(x)h(y)
\end{equation}

Reescribiéndola:

\begin{equation}
\frac{dy}{h(y)} = g(x)dx
\end{equation}

Al integrar ambos lados:

\begin{equation}
\int \frac{dy}{h(y)} = \int g(x)dx
\end{equation}

\textbf{Metodo Variable Separable}

Considere que para resolver este tipo de ejercicios, la E.D. de ser posible separarla en sus variables, para ello es necesario que este presente en la siguiente forma:


\begin{gather}
\frac{dy}{dx} \ =\ f( x,y) \ \ ( A) \ Estandar\\
\frac{dy}{dx} \ =\ A( x) *B( y) \ \ o\ A( x) /B( y) \  \notag\\
 \notag\\
 \notag\\
M( x,y) dx\ +\ N( x,y) dy\ =\ 0\ \ ( B) \ Diferencial\\
A( x) dx\ +\ B( y) dy\ =\ 0 \notag\\
\int A( x) dx\ +\ \int B( y) dy\ =\ 0 \notag\\
 \notag\\
 \notag
\end{gather}

\subsection*{Ejemplo 1 : Resolución de una ecuación separable sencilla}
Resolver \( \frac{dy}{dx} = x y \).
Separando variables:
\begin{equation}
\frac{dy}{y} = x dx
\end{equation}
Integrando:
\begin{equation}
\ln |y| = \frac{x^2}{2} + C
\end{equation}

Despejando \( y \):

\begin{equation}
y = e^{\frac{x^2}{2} + C} = Ce^{x^2/2}
\end{equation}


\subsection*{Ejemplo 2}

\begin{gather*}
\frac{dy}{dx} \ =\ -6xy,\ \ \ \ \ y( 0) \ =\ 7\\
\\
\int \frac{1}{y} \ dy\ =\ -\int 6xdx\\
ln( y) \ =\ -6\frac{x^{2}}{2} \ +c\\
Encontramos\ la\ solucion\ General\\
y\ =\ e^{\left( -3x^{2} +c\right)}\\
x^{( a+b) \ } =\ x^{( a) \ } *x^{( b) \ }\\
y\ =\ e^{-3x^{2}} e^{c} \ \ \Longrightarrow \ A\ =e^{c} \ \\
y\ =\ Ae^{-3x^{2}}\\
\\
Encontramos\ la\ solucion\ particular\ para\ y( 0) =7\\
7\ =\ Ae^{-3( 0)^{2}}\\
A\ =\ 7\\
y\ =\ 7e^{-3x^{2}}
\end{gather*}
\subsection*{Ejemplo 3}


\begin{gather*}
( 4-2x) dx-\left( 3y^{2} -5\right) dy\ =\ 0\ \ \ y( 1) =3\\
( 4-2x) dx+\left( 5-3y^{2}\right) dy\ =0\\
\int ( 4-2x) dx+\int \left( 5-3y^{2}\right) dy\ =c\\
4x-2\frac{x^{2}}{2} \ +\ 5y-3\frac{y^{3}}{3} \ =\ c\\
4x-x^{2} +5y-y^{3} \ =\ c\\
\\
Encontramos\ la\ solucion\ particular\ con\ \ y( 1) =3\\
4( 1) -( 1)^{2} +5( 3) -( 3)^{3} \ =\ c\\
c\ =\ -9\\
4x-x^{2} +5y-y^{3} \ =\ -9
\end{gather*}

\subsection{Ejercicios Adicionales}

Resolver las siguientes ecuaciones diferenciales:

\begin{enumerate}
    \item \( \tan x\cdot \sin^{2} y\ dx+\cos^{2} x\cdot \cot y\ dy=0 \), \textbf{Solución:} \( \cot^{2} y=\tan^{2} x+C \).

    \item \( xy'-y=y^{3} \), \textbf{Solución:} \( x=\frac{cy}{\sqrt{1+y^{2}}} \).

    \item \( \sqrt{1+x^{3}}\frac{dy}{dx} =x^{2} y+x^{2} \), \textbf{Solución:} \( 2\sqrt{1+x^{3}} =3\ln (y+1)+C \).

    \item \( e^{2x-y} \ dx+e^{-2x} \ dy=0 \), \textbf{Solución:} \( e^{4x} +2e^{2y} =C \).

    \item \( (x^{2} y-x^{2} +y-1)\ dx+(xy+2x-3y-6)\ dy=0 \), \textbf{Solución:} \( \frac{x^{2}}{2} +3x+y+\ln (x-3)^{10} (y-1)^{3} =C \).

    \item \( e^{x+y}\sin x\ dx+(2y+1)e^{-y^{2}} \ dy=0 \), \textbf{Solución:} \( e^{x} (\sin x-\cos x)-2e^{-y^{2}} =C \).

    \item \( 3e^{x} \cdot \tan y\ dx+(1-e^{x} )\sec^{2} y\ dy=0 \), \textbf{Solución:} \( \tan y=C(1-e^{x} )^{3} \).

    \item \( e^{y}\left(\frac{dy}{dx} +1\right) =1 \), \textbf{Solución:} \( \ln (e^{y} -1)=C-x \).

    \item \( y'=1+x+y^{2} +xy^{2} \), \textbf{Solución:} \( \arctan y-\frac{x^{2}}{2} =C \).

    \item \( y-xy'=a(1+x^{2} y) \), \textbf{Solución:} \( y=\frac{a+cx}{1+ax} \).

\end{enumerate}


\section{Ecuaciones Homogéneas}
En este tipo de ecuaciones es practico buscar la forma de representar la E.D. de la siguiente forma:
\begin{equation*}
\frac{dy}{dx} \ =\ f\left(\frac{x}{y} +o\ \ ...\frac{y}{x}\right)
\end{equation*}
Tras buscar la forma de representar la .E.D en ese esquema, se puede hacer el cambio de una de las variables de la siguiente forma
\begin{gather*}
y\ =vx\ \ \ v\ =\frac{y}{x}\\
\frac{dy}{dx} \ =\ v\ +x\frac{dv}{dx} \ \ \Longrightarrow \ dy\ =\ vdx+xdv\\
\\
o\\
x\ =\ uy\ \ \Longrightarrow \ u\ =\ \frac{x}{y}\\
\frac{dx}{dy} \ =\ u\ +y\frac{du}{dy} \ \ \ \Longrightarrow \ dx\ =udy+ydu
\end{gather*}
\textbf{¿Cómo identificar que una E.D. es homogénea?}

Dada una ecuación diferencial de la forma:
\begin{equation*}
    \frac{dy}{dx} = f(x,y)
\end{equation*}
Incorporamos el operador \(\lambda\) en la función:
\begin{equation*}
    \frac{dy}{dx} = f(\lambda x, \lambda y)
\end{equation*}
El objetivo es hacer desaparecer el operador \(\lambda\) y obtener la expresión original:
\begin{equation*}
    \frac{dy}{dx} = f(x,y)
\end{equation*}

Ahora, consideramos la ecuación diferencial en forma general:
\begin{equation*}
    M(\lambda x, \lambda y) dx + N(\lambda x, \lambda y) dy = 0
\end{equation*}
Factorizamos \(\lambda\) y verificamos que el grado sea el mismo:
\begin{equation*}
    \lambda^n M(x,y) dx + \lambda^n N(x,y) dy = 0
\end{equation*}

\subsection*{ 2.1 Ejercicio}

Resolver la ecuación diferencial:

\[
2xy\frac{dy}{dx} = 4x^{2} + 3y^{2}
\]

Dividiendo entre \( 2xy \):

\[
\frac{dy}{dx} = \frac{4x^{2} + 3y^{2}}{2xy}
\]

Verificamos si la ecuación es homogénea usando \( \lambda \):

\[
\frac{dy}{dx} = \frac{4(\lambda x)^{2} + 3(\lambda y)^{2}}{2\lambda x \lambda y}
\]

\[
\frac{dy}{dx} = \frac{4\lambda^{2} x^{2} + 3\lambda^{2} y^{2}}{2\lambda^{2} xy}
\]

\[
\frac{dy}{dx} = \frac{4x^{2} + 3y^{2}}{2xy} \quad \Rightarrow \quad \text{E.D. Original}
\]

Utilizamos el cambio de variable \( y = vx \), donde \( v = \frac{y}{x} \), y aplicamos la derivada:

\[
\frac{dy}{dx} = v + x\frac{dv}{dx}
\]

Reemplazamos en la ecuación:

\[
v + x\frac{dv}{dx} = \frac{4x^{2} + 3(vx)^{2}}{2xvx}
\]

Alternativamente, reescribimos la ecuación:

\[
\frac{dy}{dx} = \frac{4x^{2}}{2xy} + \frac{3y^{2}}{2xy}
\]

\[
\frac{dy}{dx} = 2\frac{x}{y} + \frac{3}{2}\frac{y}{x}
\]

Sustituyendo \( y = vx \):

\[
v + x\frac{dv}{dx} = 2\frac{1}{v} + \frac{3}{2} v
\]

\[
x\frac{dv}{dx} = 2\frac{1}{v} + \frac{3}{2} v - v
\]

\[
x\frac{dv}{dx} = \frac{2}{v} + \frac{v}{2}
\]

\[
x\frac{dv}{dx} = \frac{4+v^{2}}{2v} \quad \Rightarrow \quad \text{E.D. de Variables Separables}
\]

\[
\frac{2v}{4+v^{2}} dv = \frac{1}{x} dx
\]

\[
\int \frac{2v}{4+v^{2}} dv = \int \frac{1}{x} dx
\]

Asumiendo que \( c = \ln(c) \), se obtiene:

\[
\ln(4+v^{2}) = \ln(x) + \ln(c)
\]

Aplicamos propiedades logarítmicas:

\[
\ln(4+v^{2}) = \ln(cx)
\]

Aplicamos la exponencial:

\[
e^{\ln(4+v^{2})} = e^{\ln(cx)}
\]

\[
4+v^{2} = cx
\]

Reemplazamos \( v = \frac{y}{x} \):

\[
4+\frac{y^{2}}{x^{2}} = cx
\]


\subsection{Ejercicio 2.3}

Resolver la siguiente ecuación diferencial:

\[
(x^{2} +y^{2}) dx + (x^{2} -xy) dy = 0
\]

Expresamos en la forma estándar:

\[
\frac{dy}{dx} = -\frac{x^{2} +y^{2}}{x^{2} -xy}
\]

Sustituyendo \( y = vx \) con \( v = \frac{y}{x} \):

\[
\frac{dv}{dx} x + v = -\frac{x^{2} + v^{2} x^{2}}{x^{2} - x x v}
\]

Factorizando:

\[
\frac{dv}{dx} x + v = -\frac{x^{2} (1 + v^{2})}{x^{2} (1 - v)}
\]

\[
\frac{dv}{dx} x = -\frac{1 + v^{2}}{1 - v} - v
\]

\[
\frac{dv}{dx} x = \frac{- (1 + v^{2}) - v (1 - v)}{1 - v}
\]

\[
\frac{dv}{dx} x = \frac{-1 - v^{2} - v + v^{2}}{1 - v}
\]

\[
\frac{dv}{dx} x = -\frac{1 + v}{1 - v}
\]

Integramos ambos lados:

\[
\int \frac{1 - v}{1 + v} dv = -\int \frac{1}{x} dx
\]

Utilizando fracciones parciales:

\[
\int \left( \frac{1}{1+v} - \frac{v}{1+v} \right) dv = -\int \frac{1}{x} dx
\]




\subsection*{Ejercicios Propuestos}
Resuelve las siguientes ecuaciones homogéneas
\begin{enumerate}
    \item $(4x^{2} +xy-3y^{2} )\,dx+(-5x^{2} +2xy+y^{2} )\,dy=0$
    \item $\frac{dy}{dx} =\frac{y}{x} +\left(\frac{\varphi \left(\frac{y}{x}\right)}{\varphi '\left(\frac{y}{x}\right)}\right)$
    \item $xy'=2(y-\sqrt{xy})$
    \item $\left( x\cos\left(\frac{y}{x}\right) -y\right) dx+x\ dy=0$
    \item $xy'=y+2xe^{-y/x}$
    \item $dy =\left(\frac{y}{x} -\cos^{2}\left(\frac{y}{x}\right)\right) dx$
\end{enumerate}

\section*{Ecuaciones Diferenciales Exactas}

Una ecuación diferencial de la forma:
\[
M(x,y)dx + N(x,y)dy = 0
\]
es \textbf{exacta} si y solo si se cumple la siguiente condición:
\[
\frac{\partial M}{\partial y} = \frac{\partial N}{\partial x}
\]

\subsection*{Método de Solución}

Para resolver ecuaciones diferenciales exactas, se sigue el siguiente procedimiento:

\begin{enumerate}
    \item \textbf{Tomar} \( dg(x,y) = Mdx \) o bien \( dg(x,y) = Ndy \).
    \item \textbf{Integrar} en \( x \) o en \( y \).
    \item \textbf{Derivar} con respecto a la variable opuesta.
    \item \textbf{Igualar} el resultado con \( M \) o \( N \) según corresponda.
    \item \textbf{Resolver} la integral resultante.
\end{enumerate}

\subsection*{Ecuaciones No Exactas}

Cuando la ecuación diferencial no cumple con la condición de exactitud:
\[
\frac{\partial M}{\partial y} \neq \frac{\partial N}{\partial x}
\]
se debe encontrar un \textbf{factor integrante} \( F(x,y) \) que convierta la ecuación en exacta, de tal manera que:

\[
F(x,y) Mdx + F(x,y) Ndy = 0
\]

\subsection*{Cálculo del Factor Integrante}

Dependiendo de si el factor integrante es función de \( x \) o de \( y \), se calcula de la siguiente forma:

\begin{itemize}
    \item \textbf{Si el factor es función de \( x \)}:
    \[
    F(x) = e^{\int P(x) dx}, \quad \text{donde } P(x) = \frac{M_y - N_x}{N}
    \]
    
    \item \textbf{Si el factor es función de \( y \)}:
    \[
    F(y) = e^{\int P(y) dy}, \quad \text{donde } P(y) = \frac{N_x - M_y}{M}
    \]

    \item \textbf{Si el factor es función de ambas variables \( x \) y \( y \)}:  
    Se debe determinar por tanteo o con métodos específicos.
\end{itemize}

\subsection*{Resolución con el Factor Integrante}

Una vez multiplicada la ecuación diferencial por el factor integrante, la ecuación resultante se puede resolver con el método de ecuaciones diferenciales exactas.

\[
   Fi( x,y) M( x,y) dx+Fi( x,y) N( x,y) dy=0\ 
    \]

    
\subsection*{Ejemplo 2.3: Resolviendo una ecuación exacta}


\[
(4x+2y^{2})dx + (4xy)dy = 0
\]

Tomamos la ecuación en la forma:
\[
\frac{\partial g( x,y)}{\partial y} = N( x,y)
\]
Reemplazamos \( N(x,y) \):
\[
\frac{\partial g( x,y)}{\partial y} = 4xy
\]

Pasamos la derivada al otro lado como una integral:
\[
g( x,y) = \int 4xy \, dy
\]
Resolviendo la integral:
\[
g( x,y) = 2xy^{2} + c_{1} + v( x)
\]
donde \( v(x) \) es una función desconocida.

Complementamos con la ecuación:
\[
\frac{\partial g( x,y)}{\partial x} = M( x,y)
\]
Derivamos \( g(x,y) \) con respecto a \( x \):
\[
\frac{\partial}{\partial x} \left( 2xy^{2} + c_{1} + v(x) \right) = 2y^{2} + v'(x)
\]
Igualamos con \( M(x,y) \):
\[
2y^{2} + v'(x) = 4x+ 2y^{2}
\]

Despejamos \( v'(x) \):
\[
v'(x) = 4x
\]
Para encontrar \( v(x) \), integramos:
\[
v( x) = \int 4x dx
\]
\[
v( x) = 2x^{2} + c_{2}
\]

Reemplazamos en la ecuación inicial de \( g(x,y) \):
\[
g( x,y) = 2xy^{2} + c_{1} + v( x)
\]
\[
g( x,y) = 2xy^{2} + c_{1} + 2x^{2} + c_{2}
\]

Por teoría, sabemos que:
\[
g( x,y) = C
\]
Entonces:
\[
2xy^{2} + c_{1} + 2x^{2} + c_{2} = C
\]
Unificando constantes:
\[
K = C - c_{1} - c_{2}
\]
\[
2xy^{2} + 2x^{2} = K
\]

Finalmente, despejamos \( y \):
\[
y = \sqrt{\frac{K - 2x^{2}}{2x}}
\]


\subsection{Ejemplo de Resolución de Ecuaciones No Exactas}

\[
xydx + \left( 2x^{2} +3y^{2} -20\right) dy = 0
\]

\noindent Verificamos si es exacta aplicando la condición:
\[
\frac{\partial M( x,y)}{\partial y} = \frac{\partial N( x,y)}{\partial x}
\]

\noindent Calculamos las derivadas parciales:
\begin{align*}
\frac{\partial M( x,y)}{\partial y} &= \frac{\partial ( xy)}{\partial y} = x, \\
\frac{\partial N( x,y)}{\partial x} &= \frac{\partial ( 2x^{2} +3y^{2} -20)}{\partial x} = 4x.
\end{align*}

\noindent Como \( x \neq 4x \), la ecuación no es exacta. Utilizamos las fórmulas para encontrar factores de integración.

\noindent Primero intentamos con la opción (A):
\begin{align*}
\frac{1}{2x^{2} +3y^{2} -20} ( x-4x) &= \frac{-3x}{2x^{2} +3y^{2} -20}.
\end{align*}

\noindent Como el resultado no es una función exclusiva de \( x \), descartamos esta opción.

\noindent Probamos con la opción (B):
\begin{align*}
\frac{1}{xy} ( x-4x) &= -\frac{3}{y}.
\end{align*}

\noindent Como depende exclusivamente de \( y \), podemos aplicar el siguiente factor integrante:
\begin{align*}
I( x,y) &= e^{-\int h( y) dy} = e^{-\int -\frac{3}{y} dy} = e^{3\ln( y)}.
\end{align*}

\noindent Aplicando propiedades logarítmicas:
\[
I( x,y) = e^{\ln( y^{3})} = y^{3}.
\]

\noindent Multiplicamos la ecuación por el factor integrante:
\[
y^{3} ( xydx + ( 2x^{2} +3y^{2} -20) dy) = 0.
\]

\noindent Ahora verificamos si es exacta:
\begin{align*}
\frac{\partial g( x,y)}{\partial x} &= xy^{4}, \\
g( x,y) &= \int xy^{4} dx = \frac{y^{4} x^{2}}{2} + h( y).
\end{align*}

\noindent Aplicamos la segunda condición:
\begin{align*}
\frac{\partial g( x,y)}{\partial y} &= \frac{\partial}{\partial y} \left( \frac{y^{4} x^{2}}{2} + h( y)\right) = 2x^{2} y^{3} +3y^{5} -20y^{3}, \\
2x^{2} y^{3} + h'( y) &= 2x^{2} y^{3} +3y^{5} -20y^{3}, \\
h'( y) &= 3y^{5} -20y^{3}.
\end{align*}

\noindent Integrando \( h'(y) \):
\begin{align*}
h( y) &= \int ( 3y^{5} -20y^{3}) dy = \frac{1}{2} y^{6} -5y^{4} + c_{1}.
\end{align*}

\noindent Sustituyendo en \( g( x,y) \):
\[
g( x,y) = \frac{y^{4} x^{2}}{2} + \frac{1}{2} y^{6} -5y^{4} + c_{1} = K.
\]

\noindent Definiendo la constante \( c = K - c_{1} \):
\[
\frac{1}{2} x^{2} y^{4} + \frac{1}{2} y^{6} -5y^{4} = c.
\]

\subsection{Ejemplo de Resolución de Ecuaciones No Exactas}
Dada la ecuación:
\[
ydx - xdy = 0.
\]

\noindent Verificamos si es exacta:
\begin{align*}
\frac{\partial M( x,y)}{\partial y} &= 1, & \frac{\partial N( x,y)}{\partial x} &= -1.
\end{align*}

\noindent Aplicamos el factor integrante:
\begin{align*}
\frac{1}{N} ( 1+1) &= -\frac{2}{x}, \\
\frac{1}{M} ( 1+1) &= \frac{2}{y}.
\end{align*}

\noindent Aplicando el factor integrante:
\[
I( x,y) = e^{\int g( x) dx} = e^{-2\int \frac{1}{x} dx} = e^{-2\ln( x)} = x^{-2}.
\]

\noindent Multiplicamos por \( I(x,y) \):
\[
(y dx - x dy) x^{-2} = 0.
\]

\noindent Como ahora la ecuación es exacta, resolvemos:
\begin{align*}
\frac{\partial g( x,y)}{\partial x} &= M( x,y) = yx^{-2}, \\
g( x,y) &= \int yx^{-2} dx + h( x), \\
g( x,y) &= -\frac{y}{x} + h( x).
\end{align*}

\noindent Aplicamos la segunda condición:
\[
\frac{\partial g( x,y)}{\partial y} = N( x,y) = \frac{\partial ( -\frac{y}{x} + h( x))}{\partial y} = -x^{-1}.
\]

\noindent Resolviendo:
\begin{align*}
-\frac{1}{x} + h'( x) &= -\frac{1}{x}, \\
h'( x) &= 0, \\
h( x) &= c_{1}.
\end{align*}

\noindent Finalmente:
\[
g( x,y) = -\frac{y}{x} + c_{1} = K.
\]

\noindent Multiplicamos por \( -1 \):
\[
y = Ax.
\]

\section{Ejercicios Propuestos}

Resolver las siguientes ecuaciones diferenciales en caso de ser exactas:

\begin{enumerate}
    \item \( (2xy - \tan y)dx + (x^2 - x\sec^2 y)dy = 0 \)
    
    \textbf{Respuesta:} \( x^2 y - x \tan y = K \)

    \item \( (\sin x \sin y - x e^y)dy = (e^y + \cos x \cos y)dx \)
    
    \textbf{Respuesta:} \( x e^y + \cos y \sin x = K \)

    \item \( (y + y \cos xy) dx + (x + x \cos xy) dy = 0 \)
    
    \textbf{Respuesta:} \( xy + \sin xy = K \)

    \item \( \left(\frac{y}{x} + 6x\right) dx + (\ln x -2) dy = 0 \)
    
    \textbf{Respuesta:} \( y \ln x + 3x^2 - 2y = K \)

    \item \( (\cos 2y - 3x^2 y^2)dx + (\cos 2y - 2x \sin 2y - 2x^3 y)dy = 0 \)
    
    \textbf{Respuesta:} \( \frac{\sin 2y}{2} + x \cos 2y - x^3 y^2 = c \)

    \item \( e^x (x^2 e^x + e^x + xy + y)dx + (x e^x + y)dy = 0 \)
    
    \textbf{Respuesta:} \( x y e^x + \frac{y^2}{2} + \frac{e^{2x}}{4} (2x^2 - 2x + 3)x = c \)

    \item \( (1 + y^2 + xy^2)dx + (x^2 y + y + 2xy)dx = 0 \)
    
    \textbf{Respuesta:} \( 2x + y^2 (1 + x)^2 = c \)

    \item \( (3x^2 \tan y - \frac{2y^3}{x^3})dx + (x^3 \sec^2 y + 4y^3 - \frac{3y^2}{x^2})dy = 0 \)
    
    \textbf{Respuesta:} \( x^3 \tan y + y^4 + \frac{y^3}{x^2} = c \)

    \item \( (2x + \frac{x^2 + y^2}{x^2 y})dx = \left(\frac{x^2 + y^2}{xy^2}\right) dy \)
    
    \textbf{Respuesta:} \( x^3 y + x^2 - y^2 = cxy \)

    \item \( \left(\frac{\sin 2x}{y} + x\right)dx + \left(y - \frac{\sin^2 x}{y^2}\right)dy = 0 \)
    
    \textbf{Respuesta:} \( \frac{\sin^2 x}{y} + \frac{x^2 + y^2}{2} = c \)

    \item \( \left(-\frac{xy}{\sqrt{1 + x^2}} + 2xy - \frac{y}{x}\right)dx + \left(\sqrt{1 + x^2} + x^2 - \ln x\right)dy = 0 \)
    
    \textbf{Respuesta:} \( y \sqrt{1 + x^2} + x^2 y - y \ln x = c \)
\end{enumerate}



\section{Ecuación Diferencial Lineal de Primer Orden}

Dada una ecuación diferencial lineal de primer orden en su forma estándar:

\[
a_{1}( x)\frac{dy}{dx} + a_{0}( x) y = g( x)
\]

El objetivo inicial es llevarla a su forma simple. Para ello, se puede dividir la expresión entre \( a_{1}( x) \):

\[
\frac{dy}{dx} + \frac{a_{0}( x)}{a_{1}( x)} y = \frac{g( x)}{a_{1}( x)}
\]

Para contextualizar el método, se realiza un pequeño cambio de variables:

\[
P( x) = \frac{a_{0}( x)}{a_{1}( x)}, \quad Q( x) = \frac{g( x)}{a_{1}( x)}
\]

Logrando así la siguiente representación deseada:

\[
\frac{dy}{dx} + P( x) y = Q( x)
\]

Considerando que ahora la ecuación diferencial es lineal, según el método, se procede a aplicar el procedimiento adecuado para su resolución.

\subsubsection{Resolución de una Ecuación Diferencial Lineal Demostracion}

Para resolver una ecuación diferencial lineal, se sigue el siguiente procedimiento:

Identificar un \textbf{factor de integración} \( u(x) \) que nos permita multiplicar la ecuación diferencial:

\[
u( x) y' + u( x) p( x) y = u( x) q( x)
\]

El objetivo de multiplicar por \( u(x) \) es lograr que la ecuación tenga la forma de una derivada de un producto:

\[
(u y)' = u y' + u' y
\]

Comparando términos, se obtiene:

\[
u' = u( x) p( x)
\]

\[
u' = up
\]

\[
\frac{u'}{u} = p
\]

Integrando ambos lados:

\[
\int \frac{u'}{u} dx = \int p dx
\]

\[
\ln u = \int p dx
\]

\[
u = e^{\int p dx}
\]

Así, el factor de integración es:

\[
u( x) = e^{\int p( x) dx}
\]

Multiplicamos la ecuación por \( u(x) \), lo que nos permite escribirla como una derivada exacta:

\[
( u y)' = u( x) q( x)
\]

Integrando en ambos lados:

\[
u y = \int u( x) q( x) dx
\]

Finalmente, despejamos \( y(x) \):

\[
y( x) = u( x)^{-1} \left(\int u( x) q( x) dx + C\right)
\]


\subsubsection{Ejercicio 1}

Resolver la ecuación diferencial:

\[
\frac{dy}{dt} +2y = 4
\]

Identificamos los coeficientes:

\[
p(t) = 2, \quad q(t) = 4
\]

Calculamos el factor de integración:

\[
u(t) = e^{\int 2dt} = e^{2t}
\]

Aplicamos el método:

\[
y(t) = u(t)^{-1} \left(\int u(t) q(t) dt +C\right)
\]

\[
y(t) = e^{-2t} \left(\int 4e^{2t} dt +C\right)
\]

\[
y(t) = e^{-2t} \left( 4\frac{e^{2t}}{2} +C\right)
\]

\[
y(t) = 2 + Ce^{-2t}
\]

---

\subsubsection{Ejercicio 2}

Resolver la ecuación diferencial con la condición inicial \( y(1) = 4 \):

\[
x\frac{dy}{dx} + 2y = 4x^{2}
\]

Reescribimos la ecuación en su forma estándar:

\[
y' + p(x) y = q(x)
\]

Identificamos los coeficientes:

\[
p(x) = 2, \quad q(x) = 4x^2
\]

Para que la ecuación esté correctamente formulada, dividimos entre \( x \):

\[
\frac{dy}{dx} + 2\frac{1}{x} y = 4x, \quad y(1) = 4
\]

Ahora:

\[
p(x) = 2\frac{1}{x}, \quad q(x) = 4x
\]

Calculamos el factor de integración:

\[
u(x) = e^{\int p(x) dx} = e^{2\int \frac{1}{x} dx}
\]

\[
u(x) = e^{2\ln x} = e^{\ln(x^2)}
\]

\[
u(x) = x^2
\]

Multiplicamos la ecuación diferencial por \( u(x) \):

\[
x^{2} \frac{dy}{dx} + 2xy = 4x^{3}
\]

\[
(u(x) \cdot y)' = 4x^{3}
\]

Integrando:

\[
x^{2} y = 4\int x^{3} dx
\]

\[
x^{2} y = 4\frac{x^{4}}{4} +C
\]

\[
y = x^{2} + Cx^{-2}
\]

Usamos la condición inicial \( y(1) = 4 \) para encontrar \( C \):

\[
4 = 1 + C
\]

\[
C = 3
\]

\[
y = x^{2} + 3x^{-2}
\]


\subsection{Ejercicios Propuestos}

Resolver las siguientes ecuaciones diferenciales:

\begin{enumerate}
    \item \( x \tan^2 y \, dy + x \, dy = (2x^2 + \tan y)dx \), \textbf{Respuesta:} \( \tan y = x (2\sin x + c) \).

    \item \( \frac{dy}{dx} - e^x y = \frac{1}{x^2} \sin \frac{1}{x} - e^x \cos \frac{1}{x} \), \textbf{Respuesta:} \( y = \cos \frac{1}{x} + C e^{-x} \).

    \item \( x \sin \theta \, d\theta + (x^3 - 2x^2 \cos \theta + \cos \theta)dx = 0 \), \textbf{Respuesta:} \( \cos \theta = \frac{-x}{2} + C x e^{-x^2} \).

    \item \( x^2 dy + xy dx = 8x^2 \cos^2 x dx \), \textbf{Respuesta:} \( xy = 2x^2 + 2x \sin 2x + \cos 2x + c \).

    \item \( (x^5 + 3y)dx - x dy = 0 \), \textbf{Respuesta:} \( y = x^3 \left(\frac{x^2}{2} + c \right) \).

    \item \( dy = x^{-5} (4x^4 y + 3x^4 y^{-1} + 256y^7 + 768y^5 + 864y^3 + 432y + 81y^{-1})dx \).

    \item \( \frac{dy}{dx} - y \cot x = \frac{\sin(2x)}{2} \), \textbf{Respuesta:} \( y = K \sin x + \sin^2 x \).

    \item \( \cos y \, dx = (x \sin y + \tan y) dy \), \textbf{Respuesta:} \( x = K \sec y - \sec y \ln \cos y \).

\end{enumerate}


\section{Ecuación Diferencial de Bernoulli}

Las ecuaciones de Bernoulli tienen la siguiente representación. A diferencia de una ecuación diferencial lineal, estas cuentan con un término \( y^n \) multiplicando a \( q(x) \):

\[
y' + p(x) y = q(x) y^n
\]

donde \( n \) es un número real diferente de \( 0 \) y \( 1 \).

El objetivo es buscar una forma de convertir la ecuación diferencial en una ecuación lineal.

Para eliminar \( y^n \) en el miembro derecho, dividimos entre \( y^n \):

\[
y^{-n} y' + p(x) y y^{-n} = q(x)
\]

\[
y^{-n} y' + p(x) y^{1-n} = q(x)
\]

Para lograr el formato estándar de una ecuación lineal, realizamos el cambio de variable:

\[
z = y^{1-n} \quad \Rightarrow \quad \text{(A)}
\]

Reemplazando en la ecuación:

\[
y^{-n} y' + p(x) z = q(x)
\]

Buscamos una forma de expresar \( y^{-n} y' \) en términos de \( z' \), es decir, \( \frac{dz}{dx} \).

Derivamos la ecuación (A):

\[
\frac{dz}{dy} = (1 - n) y^{-n}
\]

Aplicamos la regla de la cadena para incorporar \( dx \):

\[
\frac{dz}{dy} = (1 - n) y^{-n} \frac{dx}{dx}
\]

Esto permite separar las variables:

\[
\frac{dz}{dx} = (1 - n) y^{-n} \frac{dy}{dx}
\]

Para aplicar este cambio en la ecuación original, multiplicamos por \( (1 - n) \):

\[
(1 - n) y^{-n} y' + (1 - n) p(x) z = (1 - n) q(x)
\]

\[
z' + (1 - n) p(x) z = (1 - n) q(x)
\]

Esta es una ecuación diferencial lineal en términos de \( z \), que puede resolverse con el método estándar.



\subsection* {Ejercicios 1.2}


Resolver la ecuación diferencial:

\[
2xy \frac{dy}{dx} = 4x^2 + 3y^2
\]

Reordenamos:

\[
2xy \frac{dy}{dx} - 4x^2 = 3y^2
\]

Dividimos entre \( 2xy \):

\[
\frac{dy}{dx} - 2x \frac{1}{y} = \frac{3}{2x} y
\]

Forma de Bernoulli:

\[
y' - \frac{3}{2x} y = 2x y^{-1}, \quad n = -1
\]

Realizamos el cambio de variable:

\[
z = y^{1-n} = y^2
\]

Derivamos:

\[
\frac{dz}{dx} = 2y \frac{dy}{dx}
\]

Multiplicamos por \( (1-n) = 2 \):

\[
2yy' - \frac{3}{x} y^2 = 4x
\]

\[
z' - \frac{3}{x} z = 4x
\]

Factor de integración:

\[
u(x) = e^{\int -\frac{3}{x} dx} = e^{-3\ln x} = x^{-3}
\]

Resolviendo:

\[
z(x) = x^3 \left(\int 4x^{-2} dx + C\right)
\]

\[
z(x) = x^3 \left(-4x^{-1} + C\right)
\]

Reemplazamos \( z = y^2 \):

\[
y^2 = x^3 \left(-4x^{-1} + C\right)
\]

\[
y^2 = Cx^3 - 4x^2
\]

---

\subsection* {Ejercicios 1.3}

Resolver la ecuación diferencial:

\[
x \frac{dy}{dx} + 6y = 3x y^{4/3}
\]

Dividimos entre \( x \):

\[
\frac{dy}{dx} + \frac{6}{x} y = 3y^{4/3}
\]

Dividimos entre \( y^{4/3} \), con \( n = 4/3 \):

\[
y^{-4/3} \frac{dy}{dx} + \frac{6}{x} y^{-1/3} = 3
\]

Cambio de variable:

\[
z = y^{1-n} = y^{-1/3}
\]

Derivamos y aplicamos la regla de la cadena:

\[
\frac{dz}{dx} = -\frac{1}{3} y^{-4/3} \frac{dy}{dx}
\]

Multiplicamos por \( -\frac{1}{3} \):

\[
\frac{dz}{dx} - \frac{2}{x} z = -1
\]

Ecuación diferencial lineal con:

\[
p(x) = -\frac{2}{x}, \quad q(x) = -1
\]

Factor de integración:

\[
u(x) = e^{\int -\frac{2}{x} dx} = x^{-2}
\]

Resolviendo para \( z(x) \):

\[
z(x) = x^2 \left(-\int x^{-2} dx + C\right)
\]

\[
z(x) = x^2 \left(x^{-1} + C\right)
\]

Reemplazamos \( z = y^{-1/3} \):

\[
y^{-1/3} = x^2 \left(x^{-1} + C\right)
\]

---

\subsection* {Ejercicios 1.4 - Casos especiales}

Resolver la ecuación diferencial:

\[
2x e^{2y} \frac{dy}{dx} = 3x^4 + e^{2y}
\]

Intentamos separación de variables:

\[
\frac{dy}{dx} = \frac{3x^4 + e^{2y}}{2x e^{2y}}
\]

No es posible separar, verificamos homogeneidad:

\[
\frac{dy}{dx} = \frac{3 \lambda^4 x^4 + e^{2y \lambda}}{2x \lambda e^{2y \lambda}}
\]

Forma diferencial:

\[
(3x^4 + e^{2y}) dx - 2x e^{2y} dy = 0
\]

No es exacta. Intentamos el cambio:

\[
z = e^{2y}
\]

Derivamos:

\[
\frac{dz}{dx} = 2e^{2y} \frac{dy}{dx}
\]

Sustituyendo en la ecuación:

\[
x \frac{dz}{dx} = 3x^4 + z
\]

Dividimos entre \( x \):

\[
\frac{dz}{dx} = 3x^3 + \frac{1}{x} z
\]

La ecuación se vuelve lineal:

\[
\frac{dz}{dx} - \frac{1}{x} z = 3x^3
\]

\[
p(x) = -\frac{1}{x}, \quad q(x) = 3x^3
\]

---

\subsection* {Ejercicios 1.5}

\[
x e^y \frac{dy}{dx} = 2(e^y + x^3 e^{2x})
\]

\textbf{Solución:}

\[
y = \ln \left( Cx^2 + x^2 e^{2x} \right)
\]

\subsection{Ejercicios Propuestos}

Resolver las siguientes ecuaciones diferenciales:

\begin{enumerate}
    \item \( (x^2 + y^2 + 1)dy + xy dx = 0 \), \textbf{Respuesta:} \( y^4 + 2x^2 y^2 + 2y^2 = K \).

    \item \( \frac{dy}{dx} + \frac{y}{x+1} = -\frac{1}{2}(x+1)^3 y^2 \), \textbf{Respuesta:} \( \frac{1}{y^2} = \frac{(x+1)^4}{2} + C(x+1)^2 \).

    \item \( (x^2 + 1)y' = xy + x^2 y^2 \), \textbf{Respuesta:} \( \frac{1}{y} = \frac{1}{\sqrt{1+x^2}} \left(-\frac{1}{2} \ln |x + \sqrt{1+x^2}| - x \sqrt{1+x^2} + c \right) \).

    \item \( \frac{dy}{dx} = \frac{4 \sin^2 y}{x^5 + x \tan y} \), \textbf{Respuesta:} \( x^4(K - \ln \tan y) = \tan y \).

    \item \( (x^2 + y^2 + (y + 2x)x^{-1}) dy = (2(x^2 + y^2) + (y + 2x)x^{-2}y)dx \),  
    \textbf{Respuesta:} \( (y - 2x)^2 = -\frac{2y}{x} + 10 \arctan \frac{y}{x} \), \textbf{Sugerencia:} \( y = ux \).

    \item \( (xy^2)' = (xy)^3 (x^2 + 1) \), \textbf{Respuesta:} \( y = \frac{45}{45 \sqrt{x} - 5x^5 - 9x^3} \).

    \item \( dy - y \sin x \,dx = y \ln(y e^{\cos x})dx \), \textbf{Respuesta:} \( y = -e^{x - \cos x} \).

    \item \( (x + y^3) + 6xy^2 y' = 0 \), \textbf{Respuesta:} \( y^3 = -\frac{x}{3} + C \frac{1}{2} \).

\end{enumerate}

