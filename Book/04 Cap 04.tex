\section{Introducción}
Las ecuaciones diferenciales de orden superior son fundamentales en el estudio de la dinámica de sistemas físicos, eléctricos y mecánicos. En este capítulo exploraremos métodos analíticos para resolverlas, así como sus aplicaciones.

\section{Definiciones y Propiedades Fundamentales}
Una ecuación diferencial de orden \( n \) se expresa como:

\begin{equation}
y^{(n)} + a_{n-1} y^{(n-1)} + \dots + a_1 y' + a_0 y = g(x)
\end{equation}

Si \( g(x) = 0 \), la ecuación es homogénea; si \( g(x) \neq 0 \), es no homogénea.

\section{Solución de la Ecuación Homogénea}
Para resolver:

\begin{equation}
y'' + ay' + by = 0
\end{equation}

asumimos \( y = e^{rx} \), lo que nos lleva a la ecuación característica:

\begin{equation}
r^2 + ar + b = 0
\end{equation}

Los valores de \( r \) determinan la solución.

\subsection*{Caso 1: Raíces Reales y Distintas}
Si \( r_1 \) y \( r_2 \) son distintas, la solución general es:

\begin{equation}
y_h = C_1 e^{r_1 x} + C_2 e^{r_2 x}
\end{equation}

\subsection*{Caso 2: Raíces Reales e Iguales}
Si \( r_1 = r_2 = r \), la solución es:

\begin{equation}
y_h = C_1 e^{rx} + C_2 x e^{rx}
\end{equation}

\subsection*{Caso 3: Raíces Complejas}
Si \( r = \alpha \pm i\beta \), la solución es:

\begin{equation}
y_h = e^{\alpha x} \left( C_1 \cos \beta x + C_2 \sin \beta x \right)
\end{equation}

\section{El Método de Coeficientes Indeterminados}
Se usa cuando \( g(x) \) es una combinación de polinomios, exponenciales y senoidales. La solución particular \( y_p \) se asume con una forma similar a \( g(x) \), con coeficientes por determinar.

\subsection*{Ejemplo 1: Solución para \( g(x) = e^{2x} \)}
Resolver:

\begin{equation}
y'' - 3y' + 2y = e^{2x}
\end{equation}

\textbf{Paso 1: Resolver la homogénea}
La ecuación característica es:

\begin{equation}
r^2 - 3r + 2 = 0
\end{equation}

Factorizando:

\begin{equation}
(r - 1)(r - 2) = 0 \Rightarrow r = 1, 2
\end{equation}

Solución homogénea:

\begin{equation}
y_h = C_1 e^x + C_2 e^{2x}
\end{equation}

\textbf{Paso 2: Proponer una solución particular}
Como \( g(x) = e^{2x} \), proponemos:

\begin{equation}
y_p = A x e^{2x}
\end{equation}

Derivando:

\begin{equation}
y_p' = A e^{2x} + 2Ax e^{2x}
\end{equation}

\begin{equation}
y_p'' = 2A e^{2x} + 2A e^{2x} + 4Ax e^{2x}
\end{equation}

Sustituyendo en la ecuación:

\begin{equation}
(2A e^{2x} + 2A e^{2x} + 4Ax e^{2x}) - 3(A e^{2x} + 2Ax e^{2x}) + 2Ax e^{2x} = e^{2x}
\end{equation}

Resolviendo para \( A \), obtenemos \( A = \frac{1}{4} \). Por lo tanto:

\begin{equation}
y_p = \frac{1}{4} x e^{2x}
\end{equation}

\textbf{Solución general:}

\begin{equation}
y = C_1 e^x + C_2 e^{2x} + \frac{1}{4} x e^{2x}
\end{equation}

\section{Método de Variación de Parámetros}
Se usa cuando \( g(x) \) no permite usar coeficientes indeterminados. Se asume:

\begin{equation}
y_p = u_1 y_1 + u_2 y_2
\end{equation}

donde \( y_1 \) y \( y_2 \) son soluciones de la homogénea.

\subsection*{Ejemplo 2: Resolver \( y'' + y = \tan x \)}
\textbf{Paso 1: Resolver la homogénea}

\begin{equation}
y'' + y = 0
\end{equation}

Raíces características: \( r = \pm i \), por lo que:

\begin{equation}
y_h = C_1 \cos x + C_2 \sin x
\end{equation}

\textbf{Paso 2: Aplicar variación de parámetros}
Buscamos:

\begin{equation}
y_p = u_1 \cos x + u_2 \sin x
\end{equation}

Derivadas:

\begin{equation}
y_p' = u_1' \cos x + u_2' \sin x + u_1 (-\sin x) + u_2 \cos x
\end{equation}

\begin{equation}
y_p'' = u_1' (-\sin x) + u_2' \cos x + u_1' (-\cos x) + u_2' (-\sin x) + u_1 (-\cos x) + u_2 (-\sin x)
\end{equation}

Imponiendo \( u_1' \cos x + u_2' \sin x = 0 \) y resolviendo:

\begin{equation}
u_1 = \int -\tan x \sin x dx, \quad u_2 = \int \tan x \cos x dx
\end{equation}

Tras integración, encontramos:

\begin{equation}
y_p = -\ln |\cos x| \cos x + \ln |\cos x| \sin x
\end{equation}

\textbf{Solución general:}

\begin{equation}
y = C_1 \cos x + C_2 \sin x - \ln |\cos x| \cos x + \ln |\cos x| \sin x
\end{equation}

\section{Ejercicios}
Resuelva las siguientes ecuaciones:

\begin{enumerate}
    \item \( y'' + 4y = e^x \)
    \item \( y'' - 2y' + y = x^2 \)
    \item \( y'' + y = \cos 2x \)
\end{enumerate}

