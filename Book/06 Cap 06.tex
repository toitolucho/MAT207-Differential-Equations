\section{Introducción}
Las series de potencia son herramientas fundamentales en el análisis matemático y la resolución de ecuaciones diferenciales. Permiten representar funciones como sumas infinitas y son especialmente útiles cuando la solución en términos elementales no es posible.

En este capítulo se explorarán las propiedades básicas de las series de potencia, su radio de convergencia, el desarrollo de funciones y su aplicación en la solución de ecuaciones diferenciales.

\section{Definición y Propiedades de una Serie de Potencia}
Una serie de potencia es una serie infinita de la forma:

\begin{equation}
\sum_{n=0}^{\infty} c_n (x - x_0)^n
\end{equation}

donde \( c_n \) son coeficientes constantes y \( x_0 \) es el centro de la expansión.

\subsection{Radio de Convergencia}
El radio de convergencia \( R \) de una serie de potencia se determina mediante:

\begin{equation}
R = \frac{1}{\limsup\limits_{n \to \infty} |c_n|^{1/n}}
\end{equation}

\subsection*{Ejemplo 1: Determinar el radio de convergencia}
Dada la serie:

\begin{equation}
\sum_{n=1}^{\infty} \frac{x^n}{n^2}
\end{equation}

Aplicamos la fórmula:

\begin{equation}
R = \lim_{n \to \infty} \frac{1}{(1/n^2)^{1/n}} = 1
\end{equation}

lo que indica convergencia para \( |x| < 1 \).

\section{Desarrollo de Funciones en Series de Potencia}
Se puede expresar funciones comunes en términos de series de potencia mediante sus desarrollos de Taylor y Maclaurin.

\subsection*{Ejemplo 2: Expansión de \( e^x \)}
La función exponencial puede representarse como:

\begin{equation}
e^x = \sum_{n=0}^{\infty} \frac{x^n}{n!}
\end{equation}

Expandiendo los primeros términos:

\begin{equation}
e^x = 1 + x + \frac{x^2}{2!} + \frac{x^3}{3!} + \frac{x^4}{4!} + \dots
\end{equation}

\subsection*{Ejemplo 3: Serie de Potencia igualada a \( x \)}
Supongamos la serie:

\begin{equation}
\sum_{n=1}^{\infty} \frac{x^n}{n} = x
\end{equation}

Para encontrar su radio de convergencia aplicamos:

\begin{equation}
R = \lim_{n \to \infty} \frac{1}{(1/n)^{1/n}} = 1
\end{equation}

lo que indica convergencia para \( |x| < 1 \).

\section{Solución de Ecuaciones Diferenciales mediante Series de Potencia}
Para resolver ecuaciones diferenciales en series de potencia, asumimos soluciones de la forma:

\begin{equation}
y = \sum_{n=0}^{\infty} c_n x^n
\end{equation}

Derivamos término a término y sustituimos en la ecuación dada.

\subsection*{Ejemplo 4: Resolver \( y'' - y = 0 \) usando series de potencia}
Suponemos:

\begin{equation}
y = \sum_{n=0}^{\infty} c_n x^n
\end{equation}

Derivamos dos veces:

\begin{equation}
y' = \sum_{n=1}^{\infty} n c_n x^{n-1}, \quad y'' = \sum_{n=2}^{\infty} n(n-1) c_n x^{n-2}
\end{equation}

Sustituyendo en la ecuación:

\begin{equation}
\sum_{n=2}^{\infty} n(n-1) c_n x^{n-2} - \sum_{n=0}^{\infty} c_n x^n = 0
\end{equation}

Reescribimos los índices para hacerlos coincidir:

\begin{equation}
\sum_{n=0}^{\infty} \left[ (n+2)(n+1) c_{n+2} - c_n \right] x^n = 0
\end{equation}

Para que esta ecuación sea válida para todos los valores de \( x \), los coeficientes deben cumplir:

\begin{equation}
(n+2)(n+1) c_{n+2} - c_n = 0
\end{equation}

\textbf{Paso 1: Encontrar la relación de recurrencia}

\begin{equation}
c_{n+2} = \frac{c_n}{(n+2)(n+1)}
\end{equation}

\textbf{Paso 2: Determinar los coeficientes}

Tomamos valores iniciales \( c_0 = A \) y \( c_1 = B \):

\begin{align*}
c_2 &= \frac{c_0}{2(1)} = \frac{A}{2} \\
c_3 &= \frac{c_1}{3(2)} = \frac{B}{6} \\
c_4 &= \frac{c_2}{4(3)} = \frac{A}{24} \\
c_5 &= \frac{c_3}{5(4)} = \frac{B}{120}
\end{align*}

Reconocemos la serie de Maclaurin para \( e^x \) y \( e^{-x} \), por lo que la solución es:

\begin{equation}
y = A \cosh x + B \sinh x
\end{equation}

\section{Ejercicios Resueltos}
\subsection*{Ejercicio 1: Determinar el radio de convergencia de \( \sum_{n=1}^{\infty} \frac{x^n}{n^2} \)}
\textbf{Paso 1: Aplicar la fórmula del radio de convergencia}

\begin{equation}
R = \lim_{n \to \infty} \frac{1}{(1/n^2)^{1/n}}
\end{equation}

Tomando el límite:

\begin{equation}
R = 1
\end{equation}

\subsection*{Ejercicio 2: Expandir \( \sin x \) en una serie de Taylor centrada en \( x = 0 \)}
\textbf{Paso 1: Aplicar la expansión de Taylor}

\begin{equation}
\sin x = \sum_{n=0}^{\infty} \frac{(-1)^n x^{2n+1}}{(2n+1)!}
\end{equation}

Expandiendo los primeros términos:

\begin{equation}
\sin x = x - \frac{x^3}{3!} + \frac{x^5}{5!} - \dots
\end{equation}

\section{Conclusión}
Las series de potencia permiten representar funciones y resolver ecuaciones diferenciales de forma analítica. Son herramientas esenciales en el análisis matemático y la física.

