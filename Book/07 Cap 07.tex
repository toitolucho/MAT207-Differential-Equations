
\section{Introducción}
La transformada de Laplace es una herramienta matemática fundamental en la solución de ecuaciones diferenciales. Permite convertir ecuaciones diferenciales en ecuaciones algebraicas, facilitando su resolución en aplicaciones de ingeniería, física y sistemas de control.

\section{Definición de la Transformada de Laplace}
La transformada de Laplace de una función \( f(t) \) se define como:

\begin{equation}
\mathcal{L} \{ f(t) \} = F(s) = \int_0^{\infty} e^{-st} f(t) dt, \quad \text{para } s > 0.
\end{equation}

\subsection*{Ejemplo 1: Transformada de \( f(t) = e^{at} \)}
Aplicamos la definición:

\begin{equation}
\mathcal{L} \{ e^{at} \} = \int_0^{\infty} e^{-st} e^{at} dt = \int_0^{\infty} e^{-(s-a)t} dt.
\end{equation}

Resolviendo la integral:

\begin{equation}
\mathcal{L} \{ e^{at} \} = \frac{1}{s-a}, \quad \text{para } s > a.
\end{equation}

\section{Transformadas Inversas y Transformadas de Derivadas}
\subsection{Transformadas Inversas}
La transformada inversa de Laplace se define como:

\begin{equation}
f(t) = \mathcal{L}^{-1} \{ F(s) \}.
\end{equation}

Se obtiene mediante descomposición en fracciones parciales o utilizando tablas de transformadas.

\subsection*{Ejemplo 2: Transformada inversa de \( F(s) = \frac{1}{s+3} \)}
De la tabla de transformadas sabemos que:

\begin{equation}
\mathcal{L}^{-1} \left\{ \frac{1}{s+a} \right\} = e^{-at}.
\end{equation}

Por lo tanto,

\begin{equation}
\mathcal{L}^{-1} \left\{ \frac{1}{s+3} \right\} = e^{-3t}.
\end{equation}

\subsection{Transformadas de Derivadas}
Para una función \( f(t) \):

\begin{equation}
\mathcal{L} \{ f'(t) \} = s F(s) - f(0).
\end{equation}

\begin{equation}
\mathcal{L} \{ f''(t) \} = s^2 F(s) - s f(0) - f'(0).
\end{equation}

\subsection*{Ejemplo 3: Transformada de \( y'' + 3y' + 2y = 0 \) con \( y(0) = 1 \), \( y'(0) = 0 \)}
Aplicamos la transformada de Laplace:

\begin{equation}
s^2 Y(s) - sy(0) - y'(0) + 3(sY(s) - y(0)) + 2Y(s) = 0.
\end{equation}

Sustituyendo valores iniciales:

\begin{equation}
s^2 Y(s) - s + 3s Y(s) - 3 + 2Y(s) = 0.
\end{equation}

Resolviendo para \( Y(s) \):

\begin{equation}
Y(s) = \frac{s+3}{s^2 + 3s + 2}.
\end{equation}

\section{Propiedades Operacionales I}
\subsection{Traslación en el Eje s}
Si \( \mathcal{L} \{ f(t) \} = F(s) \), entonces:

\begin{equation}
\mathcal{L} \{ e^{at} f(t) \} = F(s-a).
\end{equation}

\subsection{Traslación en el Eje t}
Si \( \mathcal{L} \{ f(t) \} = F(s) \), entonces:

\begin{equation}
\mathcal{L} \{ f(t-a) u(t-a) \} = e^{-as} F(s).
\end{equation}

\section{Propiedades Operacionales II}
\subsection{Derivadas de una Transformada}
Si \( F(s) \) es la transformada de \( f(t) \):

\begin{equation}
\mathcal{L} \{ t^n f(t) \} = (-1)^n \frac{d^n}{ds^n} F(s).
\end{equation}

\subsection{Transformadas de Integrales}
Si \( F(s) \) es la transformada de \( f(t) \):

\begin{equation}
\mathcal{L} \left\{ \int_0^t f(\tau) d\tau \right\} = \frac{1}{s} F(s).
\end{equation}

\subsection{Transformada de una Función Periódica}
Si \( f(t) \) es periódica con período \( T \), su transformada es:

\begin{equation}
\mathcal{L} \{ f(t) \} = \frac{1}{1 - e^{-sT}} \int_0^T e^{-st} f(t) dt.
\end{equation}

\section{La Función Delta de Dirac}
La función delta de Dirac \( \delta(t) \) satisface:

\begin{equation}
\int_{-\infty}^{\infty} \delta(t) f(t) dt = f(0).
\end{equation}

Su transformada de Laplace es:

\begin{equation}
\mathcal{L} \{ \delta(t - a) \} = e^{-as}.
\end{equation}

\section{Sistemas de Ecuaciones Diferenciales Lineales}
Se pueden resolver aplicando la transformada de Laplace a cada ecuación y usando álgebra matricial.

\subsection*{Ejemplo 4: Resolver \( x' = 3x + 4y \), \( y' = -x + y \) con \( x(0) = 1, y(0) = 0 \)}
Aplicamos la transformada de Laplace:

\begin{equation}
sX(s) - 1 = 3X(s) + 4Y(s),
\end{equation}

\begin{equation}
sY(s) = -X(s) + Y(s).
\end{equation}

Resolviendo el sistema en el dominio de Laplace y aplicando la transformada inversa, obtenemos:

\begin{equation}
x(t) = e^{2t} ( \cos 3t + \sin 3t), \quad y(t) = e^{2t} \sin 3t.
\end{equation}

\section{Ejercicios}
\begin{enumerate}
    \item Calcular la transformada de Laplace de \( f(t) = \cosh t \).
    \item Determinar la transformada inversa de \( F(s) = \frac{s+2}{s^2 + 4s + 5} \).
    \item Resolver \( y'' + 4y = 0 \) con \( y(0) = 0, y'(0) = 1 \) usando transformadas de Laplace.
    \item Resolver el sistema \( x' = 2x + 3y \), \( y' = -x + y \) usando Laplace.
\end{enumerate}
