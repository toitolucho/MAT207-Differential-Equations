\documentclass[12pt,a4paper]{article}
\usepackage[utf8]{inputenc}
\usepackage{amsmath,amssymb}
\usepackage{enumitem}
\usepackage{geometry}
\geometry{margin=1in}

\title{Anuncios y Puntos de Resolución de Ejercicios}
\author{}
\date{\today}

\begin{document}
\maketitle

% --- Ejercicios Homogéneos (Solutions-4.52-4.54.tex) ---

\section*{Ejercicios de Ecuaciones Diferenciales Homogéneas}

\subsection*{Ejercicio 4.52}
\textbf{Enunciado:}
\\
Resuelve la ecuación diferencial:
\[
y' = \frac{y}{x + \sqrt{xy}}
\]
\begin{itemize}[leftmargin=2em]
    \item Verifica si la ecuación es homogénea.
    \item Realiza el cambio de variable adecuado.
    \item Separa las variables y resuelve la integral.
    \item Expresa la solución general de la 
    ecuación.
    \item Encuentre la solucion particular cuando y(1) = 1
\end{itemize}

\subsection*{Ejercicio 4.54}
\textbf{Enunciado:}
\\
Resuelve la ecuación diferencial:
\[
y' = \frac{x^4 + 3x^2y^2 + y^4}{xy}
\]
\begin{itemize}[leftmargin=2em]
    \item Verifica si la ecuación es homogénea y justifica el método.
    \item Realiza el cambio de variable correspondiente.
    \item Separa las variables y resuelve la integral.
    \item Expresa la solución general de la ecuación.
    \item Encuentre la solucion particular cuando y(1) = 2
\end{itemize}

% --- Ejercicios de Modelado (Modeling-Solutions.tex) ---

\section*{Problemas de Modelado con Ecuaciones Diferenciales}

\subsection*{Problema 1: Contaminación de un Lago}
\textbf{Contexto:}
\\
En un lago, la fracción de agua contaminada $P(t)$ aumenta de manera directamente proporcional a la fracción de agua aún limpia $(1-P)$. Inicialmente el $2\%$ del agua está contaminada ($P(0) = 0.02$). Después de 4 días, el $8\%$ del agua ya se encuentra contaminada ($P(4) = 0.08$).

\textbf{Puntos a resolver:}
\begin{enumerate}[label=\textbf{\Alph*.},leftmargin=2em]
    \item Plantea la ecuación diferencial que describe el proceso.
    \item Determina la fórmula explícita que da $P(t)$ para cualquier tiempo $t$.
    \item ¿Cuál será el porcentaje de contaminación del lago después de un mes (30 días)?
    \item ¿Cuántos días deben pasar para que el lago alcance un 50\% de contaminación?
\end{enumerate}

\textbf{Análisis Crítico del Modelo:}
\begin{enumerate}[label=\textbf{E\arabic*.},leftmargin=2em]
    \item ¿En cuánto tiempo estará el lago 100\% contaminado? Justifica tu respuesta matemáticamente y explica si este resultado es completamente realista en un contexto práctico.
    \item Un nuevo informe afirma que, después de solo 5 días, la contaminación ya era del 10\%. Según el modelo que calculamos, ¿es posible esta afirmación? Explica tu razonamiento.
\end{enumerate}

\subsection*{Problema 2: Campaña de Inmunización}
\textbf{Contexto:}
\\
En una población, se inicia una campaña de inmunización cuya eficacia hace que la fracción inmunizada crezca proporcionalmente al número de personas aún no inmunizadas. Sea $P(t)$ la fracción inmunizada al tiempo $t$ (en días). Inicialmente el $10\%$ de la población está inmunizada ($P(0) = 0.10$); tras 7 días el $30\%$ lo está ($P(7) = 0.30$).

\textbf{Puntos a resolver:}
\begin{enumerate}[label=\textbf{\Alph*.},leftmargin=2em]
    \item Plantea la ecuación diferencial.
    \item Encuentra la fórmula explícita de $P(t)$.
    \item ¿Cuál será el porcentaje de la población inmunizada después de un mes (30 días)?
    \item ¿Cuántos días deben pasar para que el 80\% de la población esté inmunizada?
\end{enumerate}

\textbf{Análisis Crítico del Modelo:}
\begin{enumerate}[label=\textbf{E\arabic*.},leftmargin=2em]
    \item Según este modelo, ¿en cuánto tiempo estará el 100\% de la población inmunizada? Justifica tu respuesta matemáticamente y explica si este resultado es completamente realista en un contexto práctico.
    \item Un funcionario de salud anuncia que, después de 10 días, el 45\% de la población ya está inmunizada. Según el modelo que calculamos, ¿es posible esta afirmación? Explica tu razonamiento.
\end{enumerate}

\end{document}
