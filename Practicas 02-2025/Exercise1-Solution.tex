\documentclass[a4paper,12pt]{article}
\usepackage[utf8]{inputenc}
\usepackage{amsmath, amssymb}
\usepackage{lmodern}

\title{Resolución de Ecuación Diferencial Homogénea - Ejercicio 1}
\author{}
\date{}

\begin{document}
\maketitle

\section*{Problema}
Resolver la ecuación diferencial:
\[
y' = \frac{y-x}{x}, \qquad y(1) = 2
\]

\section*{1. Verificación de homogeneidad}
La ecuación se puede escribir como:
\[
\frac{dy}{dx} = \frac{y-x}{x} = \frac{y}{x} - 1
\]

Sustituyendo \(x \to \lambda x\), \(y \to \lambda y\):
\[
\frac{dy}{dx} = \frac{\lambda y}{\lambda x} - 1 = \frac{y}{x} - 1
\]
La ecuación es homogénea (grado 0).

\section*{2. Cambio de variable}
Sea 
\[
y = x\mu, \qquad \mu = \frac{y}{x}, \qquad \frac{dy}{dx} = \mu + x\frac{d\mu}{dx}
\]

Sustituyendo en la E.D.:
\[
\mu + x\frac{d\mu}{dx} = \mu - 1
\]

Simplificamos:
\[
x\frac{d\mu}{dx} = -1
\]

\section*{3. Separación de variables}
\[
\frac{d\mu}{dx} = -\frac{1}{x}
\]

Integrando ambos lados:
\[
\int d\mu = -\int \frac{dx}{x}
\]

\[
\mu = -\ln|x| + C
\]

\section*{4. Sustitución inversa}
Recordando que \(\mu = \frac{y}{x}\):
\[
\frac{y}{x} = -\ln|x| + C
\]

Por tanto:
\[
y = x(-\ln|x| + C) = -x\ln|x| + Cx
\]

\section*{5. Condición inicial}
Con \(y(1) = 2\):
\[
2 = -1 \cdot \ln|1| + C \cdot 1
\]

Dado que \(\ln(1) = 0\):
\[
2 = 0 + C, \qquad C = 2
\]

\section*{6. Solución particular}
\[
y = -x\ln|x| + 2x
\]

\section*{7. Solución final}
Solución general:
\[
\boxed{\, y = -x\ln|x| + Cx \,}
\]

Solución particular con \(y(1) = 2\):
\[
\boxed{\, y = -x\ln|x| + 2x \,}
\]

\end{document}
